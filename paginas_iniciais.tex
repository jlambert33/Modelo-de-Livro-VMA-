%----------------------------------------------------------------------------------------
%	TITLE PAGE
%----------------------------------------------------------------------------------------

\titlepage % Output the title page
    {\begin{tikzpicture}[remember picture,overlay]
      % Fundo (imagem ocupando toda a página)
      \node[anchor=north west,inner sep=0pt] at (current page.north west)
        {\includegraphics[width=\paperwidth]{cabecalho/output-figure0.pdf}};

      % Máscara translúcida branca no topo (ajuste opacidade/cor)
      \fill[white,opacity=0] 
        (current page.north west) rectangle (current page.south east);

      % Exemplo: moldura central ou texto

      \node[anchor=south west,inner sep=0pt] at (current page.south west)
        {\includegraphics[width=\paperwidth, angle=180]{cabecalho/output-figure0.pdf}};

    \end{tikzpicture}}%}
	%{\includegraphics[width=\paperwidth]{Images/Teste_capa.pdf}}}
        % Code to output the background image, which should be the same dimensions as the paper to fill the page entirely; leave empty for no background image
	{ % Title(s) and author(s)
		\centering\sffamily % Font styling
        \includegraphics[width=2.3cm]{./Images/logouff_vertical_azul.png}
        
		\vspace{16pt} % Vertical whitespace
		{\Huge\bfseries Modelo de livro/notas de aula\par} % Book title
		\vspace{10pt} % Vertical whitespace
		{\LARGE Dep. Matemática -- ICEx -- Volta Redonda\par} % Subtitle
        \vspace{16pt} % Vertical whitespace
  
		{\huge\bfseries Jordan Lambert \par} % Author name
        \vspace{5pt}
        {Versão: \today\par}
	}


%\listoffigures % Output the list of figures, comment or remove this command if not required

%\listoftables % Output the list of tables, comment or remove this command if not required

%----------------------------------------------------------------------------------------
%	COPYRIGHT PAGE
%----------------------------------------------------------------------------------------

\thispagestyle{empty} % Suppress headers and footers on this page

~\vfill % Push the text down to the bottom of the page

\noindent Por Jordan Lambert.

\vspace{.5cm}
\noindent 

\noindent Baseado no template de Goro Akechi, originalmente criado por Mathias Legrand, com Licença Creative Commons Atribuição NãoComercial CompartilhaIgual 4.0 Internacional (CC BY-NC-SA 4.0). Disponível em: \url{https://www.latextemplates.com/template/legrand-orange-book} % Copyright notice

\vspace{.5cm}
%\noindent \textsc{Published by Publisher}\\ % Publisher

%\noindent Repositório: \url{https://sites.google.com/view/jordanlambert/} % URL

\vspace{.5cm}
\noindent O conteúdo deste trabalho está licenciado sob a Licença Creative Commons Atribuição CompartilhaIgual 4.0 Internacional. Para ver uma cópia desta licença, visite
\url{https://creativecommons.org/licenses/by-sa/4.0/} ou envie uma carta para Creative Commons, PO Box 1866, Mountain View, CA 94042, USA. % License information, replace this with your own license (if any)

\vspace{.5cm}
\noindent \textit{Última atualização, \today} % Printing/edition date

%----------------------------------------------------------------------------------------
%	TABLE OF CONTENTS
%----------------------------------------------------------------------------------------

\pagestyle{empty} % Disable headers and footers for the following pages

\chapter*{Prefácio}

Bem-vindo ao "Modelo de Livro - VMA", um guia prático e um template elaborado em LaTeX, projetado para simplificar o processo de criação de documentos.

Este modelo nasce da necessidade de oferecer aos autores uma estrutura robusta e flexível, que combine a elegância tipográfica inerente ao LaTeX com um design moderno e acessível, inspirado na estética do "Legrand Orange Book".

Ao longo das páginas que se seguem, você encontrará não apenas um template pronto para ser preenchido, mas também um guia prático que desvenda as funcionalidades essenciais e as personalizações implementadas nesta classe de documento. Desde a estruturação de capítulos e seções até a inclusão de elementos visuais, equações matemáticas e a personalização de cores.

\setcounter{tocdepth}{0}
\tableofcontents % Output the table of contents