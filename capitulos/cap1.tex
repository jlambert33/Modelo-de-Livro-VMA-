\chapter{Um Guia Prático do modelo}

\section{A Classe do Documento}

Usamos uma classe personalizada chamada `LegrandOrangeBook`. Essa escolha afeta a estrutura padrão do documento, como a numeração de capítulos e seções, e o layout geral.

\section{Inserindo Elementos Visuais}

Imagens e tabelas são cruciais para a comunicação visual em documentos acadêmicos e técnicos.

\subsection{Imagens}

Para incluir imagens, o pacote `graphicx` é essencial. Certifique-se de que suas imagens estejam em um diretório especificado, como `Images/`, configurado no arquivo `.cls`.

\begin{verbatim}
\includegraphics[width=0.8\textwidth]{nome_da_imagem.png}
\caption{Legenda da sua imagem.}
\label{fig:minha_imagem}
\end{verbatim}

\subsection{Tabelas}

Tabelas podem ser criadas com o ambiente `tabular`. O pacote `booktabs` é recomendado para tabelas com aparência profissional.

\begin{verbatim}
\begin{table}[h!]
    \centering
    \begin{tabular}{lcc}
        \toprule
        Coluna 1 & Coluna 2 & Coluna 3 \\
        \midrule
        Dado A & Dado B & Dado C \\
        Dado D & Dado E & Dado F \\
        \bottomrule
    \end{tabular}
    \caption{Legenda da sua tabela.}
    \label{tab:minha_tabela}
\end{table}
\end{verbatim}
\begin{table}[h!]
    \centering
    \begin{tabular}{lcc}
        \toprule
        Coluna 1 & Coluna 2 & Coluna 3 \\
        \midrule
        Dado A & Dado B & Dado C \\
        Dado D & Dado E & Dado F \\
        \bottomrule
    \end{tabular}
    \caption{Legenda da sua tabela.}
    \label{tab:minha_tabela}
\end{table}

\section{Equações Matemáticas}

O LaTeX brilha na composição de equações. O ambiente `equation` é um dos mais comuns.

\begin{teo}[Exemplo de Equação]
A famosa equação de Einstein, que relaciona massa e energia, é expressa como:
\begin{equation}
E = mc^2 \label{eq:einstein}
\end{equation}
Onde $E$ é energia, $m$ é massa e $c$ é a velocidade da luz no vácuo.
\end{teo}

\section{Ambientes Personalizados do Modelo}

Este modelo de livro inclui diversos ambientes personalizados para destacar informações, como teoremas, definições, exemplos, corolários, exercícios e observações.

\subsection{Teoremas, Proposições e Corolários}
Utilize os ambientes `teo`, `prop`, `lema' e `coro` para apresentar resultados formais, proposições e corolários, respectivamente. Estes ambientes são configurados para ter um estilo visual distinto.

\begin{prop}[Afirmação Importante]
Uma proposição é uma afirmação que pode ser provada ou refutada, e é frequentemente utilizada em matemática e lógica.
\end{prop}

\begin{lema}[Afirmação Auxiliar]
Um lema é uma proposição auxiliar que serve como um passo intermediário para a prova de um teorema maior.
\end{lema}

\begin{coro}[Consequência de um Teorema]
Um corolário é uma proposição que se segue logicamente, e com pouca prova, de um teorema já provado.
\end{coro}

\subsection{Definições e Notações}
Para apresentar conceitos ou introduzir notações, use os ambientes `defic` e `notation`.

\begin{defic}[Termo Novo]
Uma definição estabelece o significado de um termo ou conceito específico no contexto do documento.
\end{defic}

\begin{notation}[Representação de um Símbolo]
O símbolo $\mathbb{R}$ representa o conjunto de todos os números reais.
\end{notation}

\subsection{Exemplos}
O ambiente `exem' é ideal para ilustrar conceitos com exemplos práticos .

\begin{exem}[Aplicação Prática]
Considere a função $f(x) = x^2 + 2x + 1$. Para $x=2$, temos $f(2) = 2^2 + 2(2) + 1 = 4 + 4 + 1 = 9$.
\end{exem}

\subsection{Comando de execício}
O comando `exer' adiciona um exercício. Se quiser, pode-se adicionar a resposta deste exercício usando em seguida o comando `resp'.

\begin{exer}
Descreva em suas próprias palavras a importância da separação de conteúdo e formatação no LaTeX.
\end{exer}
\begin{resp}
    A resposta final é 4.
\end{resp}

\begin{exer}{Exercício: Para Reflexão}
Calcule a integral
\begin{equation*}
    \int_{-\infty}^\infty e^{-x^2}dx
\end{equation*}
\end{exer}
\begin{resp}
    $\sqrt{\pi}$.
\end{resp}

Veja que a resposta não é exibida automaticamente mas é indicada que tem resposta (com o R).

Para exibir as respostas, use o comando `shipoutAnswer'.

\subsubsection{Respostas:}
\shipoutAnswer

% \begin{obs}
% Este comando para exibir as resposta pode ser usado diversas vez no texto (ao final de cada seção, capitulo, ou livro).
% \end{obs}

\begin{obs}
    Uma sugestão é usar o ambiente `secExercicios' para apresentar os exercícios. Ele cria outomaticamente a seção e coloca em duas colunas (bom para exercicios curtos pois economiza espaço).
\end{obs}
\begin{secExercicios}
\begin{exer}
    Exercicio 1
\end{exer}

\begin{exer}
    Exercicio 2
\end{exer}

\begin{exer}
    Exercicio 3
\end{exer}
\end{secExercicios}

\begin{obs}
    Estes comando apresentam eventualmente um `bug' na formatação do texto (pode pular de página indesejadamente)
\end{obs}

\subsection{Observações}
O ambiente `obs` pode ser usado para adicionar notas importantes ou observações que precisam ser destacadas.

\begin{obs}
    Esta é uma observação importante sobre o uso de ambientes personalizados. Eles ajudam a manter a consistência visual do seu documento. 
\end{obs}

\section{Personalizando Cores no Modelo}

O modelo "Legrand Orange Book" utiliza cores definidas no arquivo `main.tex'. A cor principal é escolhida no comando `cormodelo`. Para alterar a cor padrão de elementos como cabeçalhos de seções, caixas de teoremas e links, você precisará editar o comando.

\subsection{Procedimento para Alterar a Cor}
\begin{enumerate}
    \item Garanta que o Overleaf tenha uma pasta chamada `cabecalho' com algum arquivo qualquer (pode estar vazio).

    \item No arquivo `main', localize a definição `cormodelo'
    Você encontrará linhas como esta: 
    \begin{verbatim}
        \definecolor{cormodelo}{RGB}{163, 11, 11}
    \end{verbatim}
    Note que o código \{163, 11, 11\} corresponde a um tom de vermelho.

    \item Altere o código RGB da cor para a nova cor desejada. Você pode encontrar códigos de cores facilmente online. Por exemplo, para um azul vibrante, você poderia usar \{0, 0, 255\}.

    \item Para alterar a cor da figura de cabeçalho, vá ao arquivo `figura-cabecalho.tex', altere para a cor desejada o comando
    \begin{verbatim}
        \definecolor{cormodelo}{RGB}{163, 11, 11}
    \end{verbatim}
    
    Compile o arquivo `figura-cabecalho.tex' para gerar a nova imagem.

    \item Recompile seu documento `main.tex'. As mudanças de cor serão aplicadas automaticamente aos elementos que utilizam a cor.
\end{enumerate}

\begin{obs}
    Mudar o código afetará todos os elementos que a utilizam, como os títulos de capítulos, as linhas de caixas de teoremas e os links. Certifique-se de escolher uma cor que harmonize com o design geral do seu livro.
\end{obs}

\section{Adicionando Links com QR Code}

O modelo oferece um comando conveniente para adicionar links que geram um QR code automaticamente, tornando o acesso a recursos externos muito fácil para o leitor. Este recurso é implementado através do comando, `link', `video' ou `geogebra', que espera dois argumentos: a URL e uma breve descrição do link. Embora os nomes sugiram vídeo ou GeoGebra, você pode usá-los para qualquer URL, o importante é a funcionalidade de QR code.

\subsection{Comando \texttt{\string\link}}

O comando `link{URL}{Descrição do link} gera um QR code para a URL fornecida, acompanhado a descrição. 

\begin{verbatim}
\link{https://www.google.com}{Visite o Google para mais informações.}
\end{verbatim}

\link{https://www.google.com}{Visite o Google para mais informações.}

\subsection{Comando \texttt{\string\video}}

O comando `video{URL}{Descrição do link} gera um QR code para a URL fornecida, acompanhado de um ícone de vídeo e a descrição. 

\begin{verbatim}
\video{https://www.youtube.com}{Visite o Youtube para mais informações.}
\end{verbatim}

\video{https://www.google.com}{Visite o Google para mais informações.}

\subsection{Comando \texttt{\string\geogebra}}

Ó comando `geogebra{URL}{Descrição do link}' também gera um QR code, mas com um ícone de GeoGebra. 

\begin{verbatim}
\geogebra{https://www.geogebra.org/m/example}{Exemplo Interativo no GeoGebra.}
\end{verbatim}

\geogebra{https://www.geogebra.org/m/example}{Exemplo Interativo no GeoGebra.}
