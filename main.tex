%%%%%%%%%%%%%%%%%%%%%%%%%%%%%%%%%%%%%%%%%
% The Legrand Orange Book
% LaTeX Template
% Version 3.1 (February 18, 2022)
%
% This template originates from:
% https://www.LaTeXTemplates.com
%
% Authors:
% Vel (vel@latextemplates.com)
% Mathias Legrand (legrand.mathias@gmail.com)
%
% License:
% CC BY-NC-SA 4.0 (https://creativecommons.org/licenses/by-nc-sa/4.0/)
%
% Compiling this template:
% This template uses biber for its bibliography and makeindex for its index.
% When you first open the template, compile it from the command line with the 
% commands below to make sure your LaTeX distribution is configured correctly:
%
% 1) pdflatex main
% 2) makeindex main.idx -s indexstyle.ist
% 3) biber main
% 4) pdflatex main x 2
%
% After this, when you wish to update the bibliography/index use the appropriate
% command above and make sure to compile with pdflatex several times 
% afterwards to propagate your changes to the document.
%
%%%%%%%%%%%%%%%%%%%%%%%%%%%%%%%%%%%%%%%%%

%----------------------------------------------------------------------------------------
%	PACKAGES AND OTHER DOCUMENT CONFIGURATIONS
%----------------------------------------------------------------------------------------

\documentclass[
	12pt, % Default font size, select one of 10pt, 11pt or 12pt
	%fleqn, % Left align equations
	a4paper, % Paper size, use either 'a4paper' for A4 size or 'letterpaper' for US letter size
	oneside, % Uncomment for oneside mode, this doesn't start new chapters and parts on odd pages (adding an empty page if required), this mode is more suitable if the book is to be read on a screen instead of printed
]{LegrandOrangeBook}

% Book information for PDF metadata, remove/comment this block if not required 
\hypersetup{
	pdftitle={Introdução à Matemática Superior}, % Title field
	pdfauthor={Jordan Lambert}, % Author field
	pdfsubject={Apostila}, % Subject field
	pdfkeywords={IMS, Pré-Cálculo, ...}, % Keywords
	pdfcreator={LaTeX}, % Content creator field
}

\addbibresource{biblio.bib} % Bibliography file

%%%%%%%%%%%%%%%%%%%%%
%%% DEFINA A COR %%%%
\definecolor{cormodelo}{RGB}{163, 11, 11}
%%%%%%%%%%%%%%%%%%%%%

\let\cleardoublepage\clearpage

%%%%%%%%%%%%%%%%%%%%%%%%%%%%%%%%%
%   Opcões de Linguagem
%%%%%%%%%%%%%%%%%%%%%%%%%%%%%%%%%
\usepackage[english,brazil]{babel} %fazemos com que o compilador traduza expressões como “table of contents”
\usepackage{csquotes}
\usepackage[utf8]{inputenc}
\usepackage[T1]{fontenc}


%\usepackage{fancyhdr}
%\pagestyle{fancy}
%\fancyhf{}
%\fancyhead[RE]{Introdução à Matemática Superior}
%\fancyhead[LO]{\rightmark}
%\fancyhead[LE,RO]{\thepage}

%license footnote
\cfoot{\tiny{Licença \href{https://creativecommons.org/licenses/by-sa/4.0/}{CC BY-SA-4.0}.}}

%\usepackage[fixlanguage]{babelbib}
%\selectbiblanguage{brazil}

%%%% ams-latex %%%%
\usepackage{amsmath}
\usepackage{amssymb}
\usepackage{amsthm}
\usepackage{amsfonts}
\usepackage{comment}
%%%% graphics %%%%
\usepackage{graphicx} % Possibilita o uso de figuras e gráficos (suporta os formatos EPS, PDF, JPG e PNG)
%\usepackage{caption}

%%%% links %%%%
%\usepackage[colorlinks=true, allcolors=blue]{hyperref}   %usado para incluir links no texto

%%%% indent first line %%%%
\usepackage{indentfirst}

%%%% citation %%%%
%\usepackage{cite}               % usado para citações

%%%% lists %%%%
\usepackage{enumerate}          % para fazer listas com tipos diferentes de itens (a, i, 1,...)

%%%% index %%%%
%\usepackage{makeidx}            % índice remissivo

%%%% links %%%%
%\usepackage[colorlinks=true, allcolors=blue]{hyperref}   %usado para incluir links no texto

%%%% miscellaneous %%%%
\usepackage{multicol}
\usepackage{multirow}           % para tabelas, basicamente essencial
\usepackage[normalem]{ulem}     %para riscar palavra usando \sout
\usepackage{calrsfs}
\usepackage{subfigure}           %para colocar figuras lado a lado
\usepackage{xcolor}
\usepackage{tikz,tkz-base,tkz-fct}              % para criar desenhos
\usetikzlibrary{fit,shapes.geometric,arrows,shadows.blur, angles} %pra fazer os diagramas de funções
\usepackage{colortbl}
\usepackage{cleveref}

\usepackage{textcomp}
\usepackage{gensymb} %usar comando \degree
\usepackage{venndiagram}

\usepackage{shadethm}
\usepackage{bm}

\usepackage{float}
\usepackage{tasks}

\usepackage{longtable}

\setlength{\marginparwidth}{2cm}
\usepackage{todonotes}

\usepackage{polynom} %divisão de polinômios


\newcommand{\fim}{\hfill {$\Box$}}
\newcommand{\destaque}[1]{\colorbox{cormodelo!50}{$\displaystyle #1$}}
\newcommand{\iu}{{i\mkern1mu}} %unidade imaginária
\newcommand{\sen}{\operatorname{sen}}
\newcommand{\cotan}{\operatorname{cot}}
\newcommand{\cosec}{\operatorname{csc}}
\newcommand{\arcsen}{\operatorname{arcsen}}
\newcommand*\abs[1]{\left|#1\right|}

\DeclareMathOperator{\tg}{tg}
\DeclareMathOperator{\cotg}{cotg}
\DeclareMathOperator{\cossec}{cossec}
\DeclareMathOperator{\arctg}{arctg}

%\newcommand{\exemautorefname}{Exemplo}
%\newcommand*{\propautorefname}{Proposição}

\newcommand{\N}{\mathbb{N}}
\newcommand{\Z}{\mathbb{Z}}
\newcommand{\Q}{\mathbb{Q}}
\newcommand{\I}{\mathbb{I}}
\newcommand{\R}{\mathbb{R}}
\newcommand{\C}{\mathbb{C}}

\renewcommand{\geq}{\geqslant}
\renewcommand{\leq}{\leqslant}

\newcommand{\paralela}{\hspace{3pt}/\hspace{-2pt}/\hspace{3pt}}
\newcommand{\eqline}{\noalign{\smallskip} \hline \noalign{\smallskip}}

\chapterimage{cabecalho/output-figure0.pdf} % Chapter heading image
\chapterspaceabove{6.5cm} % Default whitespace from the top of the page to the chapter title on chapter pages
\chapterspacebelow{6.75cm} % Default amount of vertical whitespace from the top margin to the start of the text on chapter pages

\newenvironment{sol}
{\let\oldqedsymbol=\qedsymbol
  \renewcommand{\qedsymbol}{$\Diamond$}
  \begin{proof}[\bfseries\upshape Solução]}
  {\end{proof}
  \renewcommand{\qedsymbol}{\oldqedsymbol}}

%%%%%%%%%%%%%%%%%%%%%%%%%%%%%%
% Exercícios Resolvidos
%%%%%%%%%%%%%%%%%%%%%%%%%%%%%%
\newtheorem{exeresol}{ER}[section]
\newenvironment{resol}
{\let\oldqedsymbol=\qedsymbol
  \renewcommand{\qedsymbol}{$\Diamond$}
  \begin{proof}[\bfseries\upshape Solução]}
  {\end{proof}
  \renewcommand{\qedsymbol}{\oldqedsymbol}}
%%%%%%%%%%%%%%%%%%%%%%%%%%%%%%

% \newenvironment{dem}[1][\textbf{Demonstração:} \ ]{\textbf{#1}}{\hfill\rule{2mm}{2mm}}

%----------------------------------------------------------------------------------------

%\usetikzlibrary{external}
%\tikzexternalize[prefix=tkz/, optimize command away=\includepdf]

\begin{document}

%% CAPA
%----------------------------------------------------------------------------------------
%	TITLE PAGE
%----------------------------------------------------------------------------------------

\titlepage % Output the title page
    {\begin{tikzpicture}[remember picture,overlay]
      % Fundo (imagem ocupando toda a página)
      \node[anchor=north west,inner sep=0pt] at (current page.north west)
        {\includegraphics[width=\paperwidth]{cabecalho/output-figure0.pdf}};

      % Máscara translúcida branca no topo (ajuste opacidade/cor)
      \fill[white,opacity=0] 
        (current page.north west) rectangle (current page.south east);

      % Exemplo: moldura central ou texto

      \node[anchor=south west,inner sep=0pt] at (current page.south west)
        {\includegraphics[width=\paperwidth, angle=180]{cabecalho/output-figure0.pdf}};

    \end{tikzpicture}}%}
	%{\includegraphics[width=\paperwidth]{Images/Teste_capa.pdf}}}
        % Code to output the background image, which should be the same dimensions as the paper to fill the page entirely; leave empty for no background image
	{ % Title(s) and author(s)
		\centering\sffamily % Font styling
        \includegraphics[width=2.3cm]{./Images/logouff_vertical_azul.png}
        
		\vspace{16pt} % Vertical whitespace
		{\Huge\bfseries Modelo de livro/notas de aula\par} % Book title
		\vspace{10pt} % Vertical whitespace
		{\LARGE Dep. Matemática -- ICEx -- Volta Redonda\par} % Subtitle
        \vspace{16pt} % Vertical whitespace
  
		{\huge\bfseries Jordan Lambert \par} % Author name
        \vspace{5pt}
        {Versão: \today\par}
	}


%\listoffigures % Output the list of figures, comment or remove this command if not required

%\listoftables % Output the list of tables, comment or remove this command if not required

%----------------------------------------------------------------------------------------
%	COPYRIGHT PAGE
%----------------------------------------------------------------------------------------

\thispagestyle{empty} % Suppress headers and footers on this page

~\vfill % Push the text down to the bottom of the page

\noindent Por Jordan Lambert.

\vspace{.5cm}
\noindent 

\noindent Baseado no template de Goro Akechi, originalmente criado por Mathias Legrand, com Licença Creative Commons Atribuição NãoComercial CompartilhaIgual 4.0 Internacional (CC BY-NC-SA 4.0). Disponível em: \url{https://www.latextemplates.com/template/legrand-orange-book} % Copyright notice

\vspace{.5cm}
%\noindent \textsc{Published by Publisher}\\ % Publisher

%\noindent Repositório: \url{https://sites.google.com/view/jordanlambert/} % URL

\vspace{.5cm}
\noindent O conteúdo deste trabalho está licenciado sob a Licença Creative Commons Atribuição CompartilhaIgual 4.0 Internacional. Para ver uma cópia desta licença, visite
\url{https://creativecommons.org/licenses/by-sa/4.0/} ou envie uma carta para Creative Commons, PO Box 1866, Mountain View, CA 94042, USA. % License information, replace this with your own license (if any)

\vspace{.5cm}
\noindent \textit{Última atualização, \today} % Printing/edition date

%----------------------------------------------------------------------------------------
%	TABLE OF CONTENTS
%----------------------------------------------------------------------------------------

\pagestyle{empty} % Disable headers and footers for the following pages

\chapter*{Prefácio}

Bem-vindo ao "Modelo de Livro - VMA", um guia prático e um template elaborado em LaTeX, projetado para simplificar o processo de criação de documentos.

Este modelo nasce da necessidade de oferecer aos autores uma estrutura robusta e flexível, que combine a elegância tipográfica inerente ao LaTeX com um design moderno e acessível, inspirado na estética do "Legrand Orange Book".

Ao longo das páginas que se seguem, você encontrará não apenas um template pronto para ser preenchido, mas também um guia prático que desvenda as funcionalidades essenciais e as personalizações implementadas nesta classe de documento. Desde a estruturação de capítulos e seções até a inclusão de elementos visuais, equações matemáticas e a personalização de cores.

\setcounter{tocdepth}{0}
\tableofcontents % Output the table of contents

\pagestyle{fancy} % Enable default headers and footers again

%\cleardoublepage % Start the following content on a new page

%----------------------------------------------------------------------------------------
%	PART
%----------------------------------------------------------------------------------------

\chapterimage{cabecalho/output-figure0.pdf}
%\chapterimage{fundo_triang.png} % Chapter heading image
\chapterspaceabove{6.75cm} % Whitespace from the top of the page to the chapter title on chapter pages
\chapterspacebelow{7.25cm} % Amount of vertical whitespace from the top margin to the start of the text on chapter pages

%------------------------------------------------
\part{Início}

\chapter{Um Guia Prático do modelo}

\section{A Classe do Documento}

Usamos uma classe personalizada chamada `LegrandOrangeBook`. Essa escolha afeta a estrutura padrão do documento, como a numeração de capítulos e seções, e o layout geral.

\section{Inserindo Elementos Visuais}

Imagens e tabelas são cruciais para a comunicação visual em documentos acadêmicos e técnicos.

\subsection{Imagens}

Para incluir imagens, o pacote `graphicx` é essencial. Certifique-se de que suas imagens estejam em um diretório especificado, como `Images/`, configurado no arquivo `.cls`.

\begin{verbatim}
\includegraphics[width=0.8\textwidth]{nome_da_imagem.png}
\caption{Legenda da sua imagem.}
\label{fig:minha_imagem}
\end{verbatim}

\subsection{Tabelas}

Tabelas podem ser criadas com o ambiente `tabular`. O pacote `booktabs` é recomendado para tabelas com aparência profissional.

\begin{verbatim}
\begin{table}[h!]
    \centering
    \begin{tabular}{lcc}
        \toprule
        Coluna 1 & Coluna 2 & Coluna 3 \\
        \midrule
        Dado A & Dado B & Dado C \\
        Dado D & Dado E & Dado F \\
        \bottomrule
    \end{tabular}
    \caption{Legenda da sua tabela.}
    \label{tab:minha_tabela}
\end{table}
\end{verbatim}
\begin{table}[h!]
    \centering
    \begin{tabular}{lcc}
        \toprule
        Coluna 1 & Coluna 2 & Coluna 3 \\
        \midrule
        Dado A & Dado B & Dado C \\
        Dado D & Dado E & Dado F \\
        \bottomrule
    \end{tabular}
    \caption{Legenda da sua tabela.}
    \label{tab:minha_tabela}
\end{table}

\section{Equações Matemáticas}

O LaTeX brilha na composição de equações. O ambiente `equation` é um dos mais comuns.

\begin{teo}[Exemplo de Equação]
A famosa equação de Einstein, que relaciona massa e energia, é expressa como:
\begin{equation}
E = mc^2 \label{eq:einstein}
\end{equation}
Onde $E$ é energia, $m$ é massa e $c$ é a velocidade da luz no vácuo.
\end{teo}

\section{Ambientes Personalizados do Modelo}

Este modelo de livro inclui diversos ambientes personalizados para destacar informações, como teoremas, definições, exemplos, corolários, exercícios e observações.

\subsection{Teoremas, Proposições e Corolários}
Utilize os ambientes `teo`, `prop`, `lema' e `coro` para apresentar resultados formais, proposições e corolários, respectivamente. Estes ambientes são configurados para ter um estilo visual distinto.

\begin{prop}[Afirmação Importante]
Uma proposição é uma afirmação que pode ser provada ou refutada, e é frequentemente utilizada em matemática e lógica.
\end{prop}

\begin{lema}[Afirmação Auxiliar]
Um lema é uma proposição auxiliar que serve como um passo intermediário para a prova de um teorema maior.
\end{lema}

\begin{coro}[Consequência de um Teorema]
Um corolário é uma proposição que se segue logicamente, e com pouca prova, de um teorema já provado.
\end{coro}

\subsection{Definições e Notações}
Para apresentar conceitos ou introduzir notações, use os ambientes `defic` e `notation`.

\begin{defic}[Termo Novo]
Uma definição estabelece o significado de um termo ou conceito específico no contexto do documento.
\end{defic}

\begin{notation}[Representação de um Símbolo]
O símbolo $\mathbb{R}$ representa o conjunto de todos os números reais.
\end{notation}

\subsection{Exemplos}
O ambiente `exem' é ideal para ilustrar conceitos com exemplos práticos .

\begin{exem}[Aplicação Prática]
Considere a função $f(x) = x^2 + 2x + 1$. Para $x=2$, temos $f(2) = 2^2 + 2(2) + 1 = 4 + 4 + 1 = 9$.
\end{exem}

\subsection{Comando de execício}
O comando `exer' adiciona um exercício. Se quiser, pode-se adicionar a resposta deste exercício usando em seguida o comando `resp'.

\begin{exer}
Descreva em suas próprias palavras a importância da separação de conteúdo e formatação no LaTeX.
\end{exer}
\begin{resp}
    A resposta final é 4.
\end{resp}

\begin{exer}{Exercício: Para Reflexão}
Calcule a integral
\begin{equation*}
    \int_{-\infty}^\infty e^{-x^2}dx
\end{equation*}
\end{exer}
\begin{resp}
    $\sqrt{\pi}$.
\end{resp}

Veja que a resposta não é exibida automaticamente mas é indicada que tem resposta (com o R).

Para exibir as respostas, use o comando `shipoutAnswer'.

\subsubsection{Respostas:}
\shipoutAnswer

% \begin{obs}
% Este comando para exibir as resposta pode ser usado diversas vez no texto (ao final de cada seção, capitulo, ou livro).
% \end{obs}

\begin{obs}
    Uma sugestão é usar o ambiente `secExercicios' para apresentar os exercícios. Ele cria outomaticamente a seção e coloca em duas colunas (bom para exercicios curtos pois economiza espaço).
\end{obs}
\begin{secExercicios}
\begin{exer}
    Exercicio 1
\end{exer}

\begin{exer}
    Exercicio 2
\end{exer}

\begin{exer}
    Exercicio 3
\end{exer}
\end{secExercicios}

\begin{obs}
    Estes comando apresentam eventualmente um `bug' na formatação do texto (pode pular de página indesejadamente)
\end{obs}

\subsection{Observações}
O ambiente `obs` pode ser usado para adicionar notas importantes ou observações que precisam ser destacadas.

\begin{obs}
    Esta é uma observação importante sobre o uso de ambientes personalizados. Eles ajudam a manter a consistência visual do seu documento. 
\end{obs}

\section{Personalizando Cores no Modelo}

O modelo "Legrand Orange Book" utiliza cores definidas no arquivo `main.tex'. A cor principal é escolhida no comando `cormodelo`. Para alterar a cor padrão de elementos como cabeçalhos de seções, caixas de teoremas e links, você precisará editar o comando.

\subsection{Procedimento para Alterar a Cor}
\begin{enumerate}
    \item Garanta que o Overleaf tenha uma pasta chamada `cabecalho' com algum arquivo qualquer (pode estar vazio).

    \item No arquivo `main', localize a definição `cormodelo'
    Você encontrará linhas como esta: 
    \begin{verbatim}
        \definecolor{cormodelo}{RGB}{163, 11, 11}
    \end{verbatim}
    Note que o código \{163, 11, 11\} corresponde a um tom de vermelho.

    \item Altere o código RGB da cor para a nova cor desejada. Você pode encontrar códigos de cores facilmente online. Por exemplo, para um azul vibrante, você poderia usar \{0, 0, 255\}.

    \item Para alterar a cor da figura de cabeçalho, vá ao arquivo `figura-cabecalho.tex', altere para a cor desejada o comando
    \begin{verbatim}
        \definecolor{cormodelo}{RGB}{163, 11, 11}
    \end{verbatim}
    
    Compile o arquivo `figura-cabecalho.tex' para gerar a nova imagem.

    \item Recompile seu documento `main.tex'. As mudanças de cor serão aplicadas automaticamente aos elementos que utilizam a cor.
\end{enumerate}

\begin{obs}
    Mudar o código afetará todos os elementos que a utilizam, como os títulos de capítulos, as linhas de caixas de teoremas e os links. Certifique-se de escolher uma cor que harmonize com o design geral do seu livro.
\end{obs}

\section{Adicionando Links com QR Code}

O modelo oferece um comando conveniente para adicionar links que geram um QR code automaticamente, tornando o acesso a recursos externos muito fácil para o leitor. Este recurso é implementado através do comando, `link', `video' ou `geogebra', que espera dois argumentos: a URL e uma breve descrição do link. Embora os nomes sugiram vídeo ou GeoGebra, você pode usá-los para qualquer URL, o importante é a funcionalidade de QR code.

\subsection{Comando \texttt{\string\link}}

O comando `link{URL}{Descrição do link} gera um QR code para a URL fornecida, acompanhado a descrição. 

\begin{verbatim}
\link{https://www.google.com}{Visite o Google para mais informações.}
\end{verbatim}

\link{https://www.google.com}{Visite o Google para mais informações.}

\subsection{Comando \texttt{\string\video}}

O comando `video{URL}{Descrição do link} gera um QR code para a URL fornecida, acompanhado de um ícone de vídeo e a descrição. 

\begin{verbatim}
\video{https://www.youtube.com}{Visite o Youtube para mais informações.}
\end{verbatim}

\video{https://www.google.com}{Visite o Google para mais informações.}

\subsection{Comando \texttt{\string\geogebra}}

Ó comando `geogebra{URL}{Descrição do link}' também gera um QR code, mas com um ícone de GeoGebra. 

\begin{verbatim}
\geogebra{https://www.geogebra.org/m/example}{Exemplo Interativo no GeoGebra.}
\end{verbatim}

\geogebra{https://www.geogebra.org/m/example}{Exemplo Interativo no GeoGebra.}


%----------------------------------------------------------------------------------------

\stopcontents[part] % Manually stop the 'part' table of contents here so the previous Part page table of contents doesn't list the following chapters


%\part*{}
%\include{respostas}

%----------------------------------------------------------------------------------------
%	BIBLIOGRAPHY
%----------------------------------------------------------------------------------------

\chapterimage{} % Chapter heading image
\chapterspaceabove{2.5cm} % Whitespace from the top of the page to the chapter title on chapter pages
\chapterspacebelow{2cm} % Amount of vertical whitespace from the top margin to the start of the text on chapter pages

%------------------------------------------------

%\bibliographystyle{bababbrv}
%\bibliography{main}
%\addcontentsline{toc}{chapter}{Referências Bibliográficas}
\chapter*{Referências Bibliográficas}
\markboth{\sffamily\normalsize\bfseries Referências Bibliográficas}{\sffamily\normalsize\bfseries Referências Bibliográficas} % Set the page headers to display a Bibliography chapter name
\addcontentsline{toc}{chapter}{\textcolor{cormodelo}{Referências Bibliográficas}} % Add a Bibliography heading to the table of contents

\nocite{*}

\printbibliography[heading=bibempty]

%\include{respostas}

% \section*{Articles}
% \addcontentsline{toc}{section}{Articles} % Add the Articles subheading to the table of contents

% \printbibliography[heading=bibempty, type=article] % Output article bibliography entries

% \section*{Books}
% \addcontentsline{toc}{section}{Books} % Add the Books subheading to the table of contents

% \printbibliography[heading=bibempty, type=book] % Output book bibliography entries

%----------------------------------------------------------------------------------------
%	INDEX
%----------------------------------------------------------------------------------------

% \cleardoublepage % Make sure the index starts on an odd (right side) page
% \phantomsection
% \addcontentsline{toc}{chapter}{\textcolor{ocre}{Índice Remissivo}} % Add an Index heading to the table of contents
% \printindex % Output the index


%----------------------------------------------------------------------------------------
%	APPENDICES
%----------------------------------------------------------------------------------------

\chapterimage{cabecalho/output-figure0.pdf} % Chapter heading image
\chapterspaceabove{6.75cm} % Whitespace from the top of the page to the chapter title on chapter pages
\chapterspacebelow{7.25cm} % Amount of vertical whitespace from the top margin to the start of the text on chapter pages

%\begin{appendices*}
\appendix
\renewcommand{\chaptername}{Apêndices} % Change the chapter name to Appendix, i.e. "Appendix A: Title", instead of "Chapter A: Title" in the headers

%------------------------------------------------
\part{Apêndices}

\chapter{Apendice de exemplo}

Bla bla bla...

%----------------------------------------------------------------------------------------

\end{document}
